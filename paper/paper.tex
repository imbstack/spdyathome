\documentclass[11pt,letterpaper,notitlepage]{article}

\usepackage{fullpage}
\usepackage{enumerate}
\usepackage{amsmath}
\usepackage{amsthm}
\usepackage{paralist}
\usepackage{hyperref}
\usepackage{cite}
\usepackage{clrscode3e}
\usepackage{appendix}
\usepackage{framed}
\usepackage{graphicx}

\hypersetup{
    colorlinks,%
    citecolor=black,%
    filecolor=black,%
    linkcolor=black,%
    urlcolor=black
}

\title{An Investigation of the SPDY Protocol}
\date{\today}
\author{Brian Stack\thanks{ bis12@case.edu, Case Western Reserve University,
Cleveland, OH 44106}, Tom Dooner\thanks{ ted27@case.edu }}


\begin{document}
\maketitle
\setcounter{tocdepth}{1}
\tableofcontents
\begin{abstract}
The HTTP/1.1 specification is now well over a decade old~\cite{rfc2616}. Updates
have been constantly made to the specification, in an attempt to clear up
discrepancies or clarify murky parts of the document.  The most recent attempt
at this was made as recently as April of 2012~\cite{rfc6585}.  Even so, as the web grows
and changes, the venerable spec is beginning to show its age.  Companies such as
Google and Facebook who serve massive portions of the daily web traffic are
pushing the limit of HTTP performance, mostly due to the outdated design
assumptions made during the original specification of the protocol.  In March of
2012, httpbis (the IETF's HTTP working group) updated their charter to
include a directive of creating a new specification -- HTTP/2.0.  One of the
most likely starting points for the new protocol will be Google's
SPDY~\cite{spdyspec}.  This paper will inspect this protocol to analyze its
performance and weigh it against the design compromises it makes in order to
help decide whether or not this is a good direction for the HTTP specification
to go.
\end{abstract}

%%%%%%%%%%%%%%%%%%%%%%%%%%%%%%%%%%%%%%%%%%%%%%%%%%%%%%%%%%%%%%%%%%%%%%%%%%%%%%%
\section{Background}
\label{sec:background}
%%%%%%%%%%%%%%%%%%%%%%%%%%%%%%%%%%%%%%%%%%%%%%%%%%%%%%%%%%%%%%%%%%%%%%%%%%%%%%%
HTTP is one of the most widely used application-level protocols. Initially
written in 1996~\cite{http1rfc}, HTTP was written for a different era of web
sites -- before streaming video websites, before asset-heavy web pages, and
before mostly-Javascript web applications like Google's Gmail, Twitter, and
Facebook.

Although HTTP was amended in 1999 to include better support for caching, the
ability to reuse connections, and other helpful features, the World Wide Web
landscape has changed significantly since then -- led by big technology
companies desiring to optimize application speed and user experience.

The IETF's httpbis working group is responsible for maintaining and developing
the HTTP specification, including all aspects pertaining to developing a new
HTTP specification, HTTP/2.0. The most recent httpbis working group charter
mentions the following goals for HTTP/2.0~\cite{httpbis-charter}:
\begin{itemize}
\item Significantly improve performance in common use cases
\item Make more efficient use of network resources
\item Must be deployable on the current internet
\item Maintain ease of deployment
\item Updated security to modern standards
\end{itemize}

The most important part of the goals that is not explicitly mentioned is that
these proposals must maintain semantic equivalence with the currently
implemented HTTP specification; changes may only be made to the method in
which these request/response pairs are transmitted.

The current self-imposed deadline for HTTP/2.0 is July
2013~\cite{httpbis-charter}.  So far, three likely starting points for
the specification have been proposed: Google's SPDY, Microsoft's Speed +
Mobility, and a "Network-Friendly" proposal from a handful of Open Source
authors. The three will be briefly outlined beginning in
\S\ref{sec:background/spdy} and then the remainder of the paper will focus on
measurements of the most viable candidate for HTTP/2.0, SPDY.

% TODO: Expand on these for the full paper!
%%%%%%%%%%%%%%%%%%%%%%%%%%%%%%%%%%%%%%%%%%%%%%%%%%%%%%%%%%%%%%%
\subsection{SPDY}
\label{sec:background/spdy}
%%%%%%%%%%%%%%%%%%%%%%%%%%%%%%%%%%%%%%%%%%%%%%%%%%%%%%%%%%%%%%%
The most interesting and likely to succeed proposal, SPDY is rather mature
at this point in time, having been implemented in both the most recent Chrome
and Firefox browsers in addition to having modules and patches for both the
Apache and Nginx web servers which allow them to speak SPDY.

This protocol springs from a rather unique event in the history of the web.
Google is positioned such that they control a large portion of both the browser
and website markets simultaneously. This has allowed them to define their own
protocol without going through the traditionally slow IETF process. This is
notable because it gives empiricists the ability to measure real-world
performance metrics whilst implementing websites may benefit from the new
protocol in the short term.

The most important parts of SPDY as of draft 3 (most recent complete version
\cite{spdy3}) are framing and streams.

\begin{description}
\item[Framing Layer] This binary protocol provides a layer on top of a reliable transport
protocol (recommended to be TCP) and provides a way for HTTP request/response
pairs to be issued and received out of order over a single transport layer
connection.  This provides an expected performance improvement of HTTP/1.1 by
not requiring multiple connections to the host to be made in order to
parallelize resource acquisition. There are 2 types of frames, first is control
frames which are responsible for managing the lifecycle of a stream, which we
will define soon.  Next is the data frame which is used simply to transmit data
also allowing for streaming.
\item[Streaming] Streams are bidirectional and are used to send out of order
responses to requests. Importantly, streams can have priorities set by the
creator of the stream.  This, in conjunction with server-push (defined next)
allows for more important assets to be sent over the wires first.  An example of
the usefulness of this feature is a webpage with many images that are
unimportant for the actual content and layout of the page.  In that case, the
images could be given a low priority and sent last.
\item[Server-Push] Once a server receives a request for an asset, if it is aware
of closely connected assets, it may send them behind the original response,
without waiting for the client to request them.  This will pre-populate the
cache of the client and save a round trip time for each item.
\item[Header Compression with Dictionary] If there are many small assets on a
webpage, then a significant portion of the transmitted bytes will be due to HTTP
headers rather than the actual content.  If this can be compressed, much
bandwidth is saved~\cite{binoy}.
\end{description}

%%%%%%%%%%%%%%%%%%%%%%%%%%%%%%%%%%%%%%%%%%%%%%%%%%%%%%%%%%%%%%%
\subsection{Speed+Mobility}
\label{sec:background/s+m}
%%%%%%%%%%%%%%%%%%%%%%%%%%%%%%%%%%%%%%%%%%%%%%%%%%%%%%%%%%%%%%%
This proposal~\cite{sm} comes from Microsoft.  It shares much in common with
SPDY, but with a few major differences. First is that rather than defining
their own session layer, Microsoft S+M relies on WebSockets for a session
layer.  In addition (this is very important), S+M uses HTTP to negotiate the
protocol upgrade rather than SPDY's use of SSL/TLS to do this same thing. This
adds quite a bit of latency to beginning the actual communication and is one of
the reasons that Facebook endorsed SPDY~\cite{fbook}.

%%%%%%%%%%%%%%%%%%%%%%%%%%%%%%%%%%%%%%%%%%%%%%%%%%%%%%%%%%%%%%%
\subsection{Network-Friendly}
\label{sec:background/opensource}
%%%%%%%%%%%%%%%%%%%%%%%%%%%%%%%%%%%%%%%%%%%%%%%%%%%%%%%%%%%%%%%
This proposal comes from a group of open-source developers without any strict
ties to a company or organization otherwise.  It is quite different from the
other two proposals because it is looking at the problem from another angle.
The four main ideas in this proposal are~\cite{friendly}
\begin{itemize}
\item Binary encoding of header fields\footnote{This might be expected
considering that one of the primary authors of this proposal is the creator of
the Varnish Cache \url{https://www.varnish-cache.org/} which must inspect the
header fields of incoming requests.}
\item Grouping of common header fields allowing for reduced header size when
repeated requests are made.
\item Multiplexing of request and response.  This is something shared in common
with all three of the currently viable proposals.
\item A layering model that is easier for intermediaries like Varnish to parse.
\end{itemize}

Overall, this proposal is aimed at making it easier for network appliances such
as HTTP reverse-proxies to inspect and route incoming HTTP traffic.

Together these make for a promising platform for the next version of HTTP.  As
stated earlier, this is the most likely starting point for the new protocol. It
is important to note that currently SPDY requires SSL/TLS to upgrade the
connection and it is likely that in the future SSL/TLS will be explicitly
required.

%%%%%%%%%%%%%%%%%%%%%%%%%%%%%%%%%%%%%%%%%%%%%%%%%%%%%%%%%%%%%%%%%%%%%%%%%%%%%%%
\section{Research Areas}
\label{sec:research}
%%%%%%%%%%%%%%%%%%%%%%%%%%%%%%%%%%%%%%%%%%%%%%%%%%%%%%%%%%%%%%%%%%%%%%%%%%%%%%%
The web has developed to the point where HTTP/1.1 is no longer sufficient to
architect our major web applications. The basic semantic meaning is excellent,
and is the foundation of the web, but certain engineering considerations must
be taken into account in order to improve the protocol. For example, Google has
made heavy investments in SPDY, but the question remains whether or not this
will be an improvement for the vast majority of the web.  Does the rather
drastic increase in complexity pay off in the general case, and is it worth the
added cost in terms of development time to support this protocol everywhere?

%%%%%%%%%%%%%%%%%%%%%%%%%%%%%%%%%%%%%%%%%%%%%%%%%%%%%%%%%%%%%%%%%%%%%%%%%%%%%%%
\section{Overview}
\label{sec:research/overview}
%%%%%%%%%%%%%%%%%%%%%%%%%%%%%%%%%%%%%%%%%%%%%%%%%%%%%%%%%%%%%%%%%%%%%%%%%%%%%%%
This project sets out to answer that question in two fashions. First, we will
perform an in-depth examination of what is the limiting factor of current
HTTP/1.1 including a review of literature in the area and a characterization of
real-world web page asset size distribution in \S\ref{sec:assets} (and
how we predict the same assets will load if loaded over SPDY). Since SPDY can
combine asset requests into a single multiplexed TCP connection, we identify
this as a significant source of performance improvement over HTTP. We will
perform this by collecting the assets from the top Alexa sites with the Web
Asset Scraper described in \S\ref{sec:research/scraper}.

But how will \textit{real} web sites perform when their assets are multiplexed?
Since only a handful of (already optimized) sites implement SPDY, we must
collect direct measurements with a simulated browser. The simulated
client/server will experimentally determine empirically how much of an
improvement is seen when using SPDY rather than HTTP.  The simulation is
proposed in \S\ref{sec:research/client-server}.

%%%%%%%%%%%%%%%%%%%%%%%%%%%%%%%%%%%%%%%%%%%%%%%%%%%%%%%%%%%%%%%
\subsection{Web Asset Scraper}
\label{sec:research/scraper}
%%%%%%%%%%%%%%%%%%%%%%%%%%%%%%%%%%%%%%%%%%%%%%%%%%%%%%%%%%%%%%%
The server described in \S\ref{sec:project/server} needs to correctly simulate
real world webpages. In order to facilitate this, websites from the Alexa
rankings\footnote{\url{http://www.alexa.com}} have been scraped and the size and
number of assets on each recorded.  This database will be accessed by the server
upon each request in order to determine which data to return.  Great
consideration was taken to make sure that this data accurately represents the
state of the web, including executing javascript from each page in a headless
browser in order to find dynamically loaded assets\footnote{Thanks to Tom Dooner
for doing much implementation of this part of the collecting}. This is not part
of the proposed project as the work has already been completed in conjunction
with another student.

%%%%%%%%%%%%%%%%%%%%%%%%%%%%%%%%%%%%%%%%%%%%%%%%%%%%%%%%%%%%%%%
\subsection{Client/Server Implementation}
\label{sec:research/client-server}
%%%%%%%%%%%%%%%%%%%%%%%%%%%%%%%%%%%%%%%%%%%%%%%%%%%%%%%%%%%%%%%
The simulation uses a client-server architecture to gather data about the
effectiveness of the SPDY protocol. The server can be run from any standard
machine, but only a single one will be run for the experiment.  The client (the
``collector'') is the real measurement platform and will be distributed to
volunteers to run in order to test SPDY in a number of different network
conditions.  The client will collect data and report back to the server
anonymous information about how performant the protocol was in addition to
helpful details that may help break down what caused variations in results.
%%%%%%%%%%%%%%%%%%%%%%%%%%%%%%%%%%%%%%%%%%%%%%%%%%%%
\subsubsection{Server}
\label{sec:research/server}
%%%%%%%%%%%%%%%%%%%%%%%%%%%%%%%%%%%%%%%%%%%%%%%%%%%%
If each client accessed all of the websites we scraped, the process would take
an insufferably long time.  Therefore, we have decided on the following
breakdown
\begin{center}
\begin{tabular}{ccc}
$1-100$ top sites & 10\% of $100-1000$ sites & 1\% of $1000-10000$ sites \\
\end{tabular}
\end{center}

These sites will be selected randomly and made into a list for the client to
consume. This is a total of 280 sites that run the gamut from representing the
largest and most frequented sites on the web, down to some of the most obscure.

The client can access the server using standard RESTful paths in two fashions
\begin{itemize}
\item \texttt{/site/<n>} where \texttt{n} is an integer that specifies the index
of the website simulation that is being accessed.  This will result in the
server returning fully formed HTML containing assets that map to the proper size
of the assets from the scraped page.  The rest of the document will be filled
with random bytes until it is the proper size of the recorded webpage that was
scraped.
\item \texttt{/img/<m>} where \texttt{m} is an integer specifying the size of the
asset that should be downloaded. So, in the case that we found a 2 MB image on a
webpage that is being simulated, \texttt{m} will be 2,000,000 and the server
will return a binary blob of that size. It is possible that this could take into
account the different compressibility of text to images.  For instance, an asset
that was javascript should have somewhat less random data supplied so that it
may compress more readily as compared to an image which doesn't compress very
much on top of its generally pre-compressed format.
\end{itemize}

%%%%%%%%%%%%%%%%%%%%%%%%%%%%%%%%%%%%%%%%%%%%%%%%%%%%
\subsubsection{Client}
\label{sec:research/client}
%%%%%%%%%%%%%%%%%%%%%%%%%%%%%%%%%%%%%%%%%%%%%%%%%%%%
The client should be cross platform and easy to run.  The ultimate goal is for
it to be as simple as downloading and clicking on an executable of some form.
The client will access each of the tests specified in the list in turn using
both SPDY and HTTP/1.1 with SSL/TLS and
record timing information about how long it takes for each request to complete.
After this step is complete, the client will gather data about its location and
other considerations in order to return to the server for analysis.  All data
returned will be presented to the user to approve \textbf{before} being sent
back.

%%%%%%%%%%%%%%%%%%%%%%%%%%%%%%%%%%%%%%%%%%%%%%%%%%%%%%%%%%%%%%%%%%%%%%%%%%%%%%%
\section{Asset Characterization}
\label{sec:assets}
%%%%%%%%%%%%%%%%%%%%%%%%%%%%%%%%%%%%%%%%%%%%%%%%%%%%%%%%%%%%%%%%%%%%%%%%%%%%%%%
here is where we will discuss the asset measurement stuff.

%%%%%%%%%%%%%%%%%%%%%%%%%%%%%%%%%%%%%%%%%%%%%%%%%%%%%%%%%%%%%%%%%%%%%%%%%%%%%%%
\section{Real World SPDY vs. HTTP}
\label{sec:realworld}
%%%%%%%%%%%%%%%%%%%%%%%%%%%%%%%%%%%%%%%%%%%%%%%%%%%%%%%%%%%%%%%%%%%%%%%%%%%%%%%
here is where we will discuss brian's spdyathome results


% I don't want to remove this wholesale but I don't think it is relevant
% anywhere else in the paper. Maybe it fits somewhere?
% -Tom
% %%%%%%%%%%%%%%%%%%%%%%%%%%%%%%%%%%%%%%%%%%%%%%%%%%%%%%%%%%%%%%%
% \section{Evaluation}
% \label{sec:eval}
% %%%%%%%%%%%%%%%%%%%%%%%%%%%%%%%%%%%%%%%%%%%%%%%%%%%%%%%%%%%%%%%
% Once this data is collected the time difference will be found between each
% protocol's retrieval of each test page. The obvious next step will be to
% determine if SPDY actually made the pages load faster or not. In all likelihood,
% there will not be a constant speedup across all pages and so the next step of
% evaluation will be to see in which cases an improvement was found. This should
% give insight into how SPDY improves over the prior specification and help decide
% whether or not it is worth it for the actual upgrade of HTTP to take place.
%
% This success of this experiment will be based off of how many points of data can
% be acquired and how varying the conditions are for them.  Much of what will be
% included in the final report will be plots and tables of the data collected to
% allow the reader to decide for themselves.  This information will be useful to
% others and the measurement platform itself should be designed in such a way that
% future experiments can be run on it without too much effort to retool it.

\bibliographystyle{plain}
\bibliography{rfc,citations}

\end{document}
